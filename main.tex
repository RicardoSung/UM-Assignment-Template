%%%%%%%%%%%%%%%%%%%%%%%%%%%%%%%%%%%%%%%%%%%%%%%%%%%%%%%%%%%%%%%%%%
%%%%%%%%%%%%%%%%%%%%%%%%%%%%%%%%%%%%%%%%%%%%%%%%%%%%%%%%%%%%%%%%%%
%Packages
\documentclass[10pt, a4paper]{article}
\usepackage[top=3cm, bottom=4cm, left=3.5cm, right=3.5cm]{geometry}
\usepackage{amsmath,amsthm,amsfonts,amssymb,amscd, fancyhdr, color, comment, graphicx, environ}
\usepackage{float}
\usepackage{mathrsfs}
\usepackage[math-style=ISO]{unicode-math}
\setmainfont{TeX Gyre Termes Math}
\usepackage{lastpage}
\usepackage[dvipsnames]{xcolor}
\usepackage[framemethod=TikZ]{mdframed}
\usepackage{enumerate}
\usepackage[shortlabels]{enumitem}
\usepackage{fancyhdr}
\usepackage{indentfirst}
\usepackage{listings}
\usepackage{sectsty}
\usepackage{thmtools}
\usepackage{shadethm}
\usepackage{hyperref}
\usepackage{setspace}
\hypersetup{
    colorlinks=true,
    linkcolor=blue,
    filecolor=magenta,      
    urlcolor=blue,
}
%%%%%%%%%%%%%%%%%%%%%%%%%%%%%%%%%%%%%%%%%%%%%%%%%%%%%%%%%%%%%%%%%%
%%%%%%%%%%%%%%%%%%%%%%%%%%%%%%%%%%%%%%%%%%%%%%%%%%%%%%%%%%%%%%%%%%
%Environment setup
\mdfsetup{skipabove=\topskip,skipbelow=\topskip}
\newrobustcmd\ExampleText{%
An \textit{inhomogeneous linear} differential equation has the form
\begin{align}
L[v ] = f,
\end{align}
where $L$ is a linear differential operator, $v$ is the dependent
variable, and $f$ is a given non−zero function of the independent
variables alone.
}

\mdtheorem[style=theoremstyle]{Problem}{Problem}
\newenvironment{Solution}{\textbf{Solution.}}

%%%%%%%%%%%%%%%%%%%%%%%%%%%%%%%%%%%%%%%%%%%%%%%%%%%%%%%%%%%%%%%%%%
%%%%%%%%%%%%%%%%%%%%%%%%%%%%%%%%%%%%%%%%%%%%%%%%%%%%%%%%%%%%%%%%%%
%Fill in the appropriate information below
\newcommand{\norm}[1]{\left\lVert#1\right\rVert}     
\newcommand\course{Course Number\\Course Name}                      % <-- course name   
\newcommand\hwnumber{Homework Number}                         % <-- homework number
\newcommand\Information{Your Name\\Student Number}           % <-- personal information
%%%%%%%%%%%%%%%%%%%%%%%%%%%%%%%%%%%%%%%%%%%%%%%%%%%%%%%%%%%%%%%%%%
%%%%%%%%%%%%%%%%%%%%%%%%%%%%%%%%%%%%%%%%%%%%%%%%%%%%%%%%%%%%%%%%%%
%Page setup
\pagestyle{fancy}
\headheight 35pt
\lhead{\today}
\rhead{\includegraphics[height=1.5cm]{amsv.png}} 
% <-- school logo(please upload the file first, then change the name here)
\lfoot{}
\pagenumbering{arabic}
\cfoot{\small\thepage}
\rfoot{}
\headsep 1.2em
\renewcommand{\baselinestretch}{1.25}       
\mdfdefinestyle{theoremstyle}{%
linecolor=black,linewidth=1pt,%
frametitlerule=true,%
frametitlebackgroundcolor=gray!20,
innertopmargin=\topskip,
}
%%%%%%%%%%%%%%%%%%%%%%%%%%%%%%%%%%%%%%%%%%%%%%%%%%%%%%%%%%%%%%%%%%
%%%%%%%%%%%%%%%%%%%%%%%%%%%%%%%%%%%%%%%%%%%%%%%%%%%%%%%%%%%%%%%%%%
%Add new commands here
\renewcommand{\labelenumi}{\alph{enumi})}
\newcommand{\Z}{\mathbb Z}
\newcommand{\R}{\mathbb R}
\newcommand{\Q}{\mathbb Q}
\newcommand{\NN}{\mathbb N}
\DeclareMathOperator{\Mod}{Mod} 
\renewcommand\lstlistingname{Algorithm}
\renewcommand\lstlistlistingname{Algorithms}
\def\lstlistingautorefname{Alg.}
%%%%%%%%%%%%%%%%%%%%%%%%%%%%%%%%%%%%%%%%%%%%%%%%%%%%%%%%%%%%%%%%%%
%%%%%%%%%%%%%%%%%%%%%%%%%%%%%%%%%%%%%%%%%%%%%%%%%%%%%%%%%%%%%%%%%%
%Begin now!

\begin{document}

\begin{titlepage}
    \begin{center}
        \vspace*{3cm}
            
        \Huge
         \hwnumber  
        
            
        \vspace{1.5cm}
        \Large
            
        \Information
            
        \vfill
        
        \course \ 
            
        \vspace{1.5cm}
            
        \includegraphics[width=0.4\textwidth]{um_badge.png}
        \\
        
        \Large
        
        \today
            
    \end{center}
\end{titlepage}

%%%%%%%%%%%%%%%%%%%%%%%%%%%%%%%%%%%%%%%%%%%%%%%%%%%%%%%%%%%%%%%%%%
%%%%%%%%%%%%%%%%%%%%%%%%%%%%%%%%%%%%%%%%%%%%%%%%%%%%%%%%%%%%%%%%%%
%Start the assignment now

%%%%%%%%%%%%%%%%%%%%%%%%%%%%%%%%%%%%%%%%%%%%%%%%%%%%%%%%%%%%%%%%%%
%New problem
\newpage
\begin{Problem}
    \textbf{Smoothness} 
    
    A differential function $f$ is said to be $L$-smooth if its gradietns are Lipschitz continuous, that is
    
    $$
    \|\nabla f(x)-\nabla f(y)\| \leq L\|x-y\|
    $$
    
    let $f : \mathbb{R}^{d} \rightarrow \mathbb{R}$ be a twice differentiable function. If $f$ is $L$-smooth then prove the following inequality:
    
    \begin{itemize}
    	\item (15 pt) Prove $\left\langle\nabla^2 f(x) v, v\right\rangle \leq L\|v\|_2^2, \quad \forall x, v \in \mathbb{R}^d$
    	\item (15 pt) Prove	$f(y) \leq f(x)+\langle\nabla f(x), y-x\rangle+\frac{L}{2}\|y-x\|_2^2$
    \end{itemize}
\end{Problem}
    
\begin{Solution}
	\vspace{4pt}
	\begin{itemize}
		\item $\left\langle\nabla^2 f(x) v, v\right\rangle \leq L\|v\|_2^2, \quad \forall x, v \in \mathbb{R}^d$
		
		\textbf{Proof.}
	
		Before proving the original inequality, we firstly prove the hessian matrix of the $L$-smooth differential function is negative semi-definite.
		
		$$
		\begin{aligned}
			&\because ||\nabla f(x) - \nabla f(y)|| \leq L||x-y|| \\
			&\therefore ||x-y||\cdot||\nabla f(x) - \nabla f(y)|| \leq L||x-y||^2 \\
			&\because \langle(x-y), \nabla f(x) - \nabla f(y)\rangle \leq ||x-y||\cdot||\nabla f(x) - \nabla f(y)|| (\text{ Cauchy-Schwartz Inequality})\\
			&\text{Rearranging the above terms, we would have: } \langle Lx- \nabla f(x)+ Ly - \nabla f(y) , (x-y)\rangle	\geq 0\\
		\end{aligned} 
		$$
		
		......
				
		\item $f(y) \leq f(x)+\langle\nabla f(x), y-x\rangle+\frac{L}{2}\|y-x\|_2^2$
		
		\textbf{Proof.}
		
		The original inequality can be written as:
		$$
		f(y) - f(x)-\nabla f(x)^{T}(y-x) \leq \frac{L}{2}\|y-x\|_2^2
		$$
		
		Observing the left hand side of the inequality, let $z(t)=x+t(y-x)$ and $g(t)=f(z(t))$. Then by the Newton-Leibniz formula:
		\\
		......
		
	\end{itemize}
\end{Solution} 

\newpage
\begin{Problem}
	\textbf{Gradient descent rate with line search in strongly convex
		function} 
	
	Suppose the function $f : \mathbb{R}^{n} \rightarrow \mathbb{R}$ is strongly convex and twice differentiable, \textit{i.e} $\nabla^2 f(x) \succeq l I$ with constant $l > 0$.
	Also, its gradient is Lipschitz continuous with constant $L > 0$, \textit{i.e.} we have that $∥∇f (x) − ∇f (y)∥ ≤ L∥x − y∥$
	for any $x, y$.
	\\......

\end{Problem}

\begin{Solution}
	\\
	......	


\end{Solution} 

%%%%%%%%%%%%%%%%%%%%%%%%%%%%%%%%%%%%%%%%%%%%%%%%%%%%%%%%%%%%%%%%%%
%Complete the assignment now
\end{document}

%%%%%%%%%%%%%%%%%%%%%%%%%%%%%%%%%%%%%%%%%%%%%%%%%%%%%%%%%%%%%%%%%%
%%%%%%%%%%%%%%%%%%%%%%%%%%%%%%%%%%%%%%%%%%%%%%%%%%%%%%%%%%%%%%%%%%
